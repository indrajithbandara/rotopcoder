\begin{problem}{Descompune}
{descompune.in}{descompune.out}
{2.5 secunde} {16 megabytes (2 megabytes stiv\u{a})}{}

Marinel e pasionat de istorie. \^{I}ntr-o zi, el a g\u{a}sit o propozi\c{t}ie $P$ din care au fost eliminate spa\c{t}iile. Se presupune c\u{a} aceasta de\c{t}ine secretul dispari\c{t}iei maya\c{s}ilor, \^{i}ns\u{a} Marinel trebuie mai \^{i}nt\^{a}i s\u{a} o descompun\u{a} \^{i}n cuvinte folosind dic\c{t}ionarul maya\c{s}. \^{I}n acest dic\c{t}ionar special, fiecare cuv\^{a}nt are asociat un cost $COST(i)$.

Ajuta\c{t}i-l pe Marinel s\u{a} g\u{a}seasc\u{a} o descompunere valid\u{a} de cost minim a propozi\c{t}iei folosind cuvintele din dic\c{t}ionar precum \c{s}i costul minim al acelei descompuneri. 

\textbf{Observatie}: dac\u{a} elimin\u{a}m spa\c{t}iile dintre cuvintele din descompunere, atunci \c{s}irul de caractere rezultat trebuie s\u{a} fie identic cu $P$.


\InputFile

Pe prima linie a fi\c{s}ierului de intrare $descompune.in$ se afla un \c{s}ir de caractere $P$ f\u{a}r\u{a} spa\c{t}ii ce reprezint\u{a} propozi\c{t}ia care va trebui descompus\u{a}. Pe a doua linie va fi un num\u{a}r $N$, ce reprezint\u{a} num\u{a}rul de cuvinte din dic\c{t}ionar urmat de $N$ linii astfel: linia $i + 2$ contine un cuv\^{a}nt $CUV(i)$ urmat de un num\u{a}r natural $COST(i)$ ce reprezint\u{a} costul asociat cuv\^{a}ntului $i$.

\OutputFile

Pe prima linie a fi\c{s}ierului de ie\c{s}ire $descompune.out$, se va afi\c{s}a o descompunere valid\u{a} de cost minim a propozi\c{t}iei folosind cuvintele din dic\c{t}ionar.
Pe a doua linie se va afi\c{s}a costul minim al acelei descompuneri.
\^{I}n cazul \^{i}n care nu exist\u{a} nici o descompunere valid\u{a}, se va afi\c{s}a $-1$.

\Constraints
\begin{itemize}
	\setlength{\itemsep}{1pt}
  	\setlength{\parskip}{0pt}
  	\setlength{\parsep}{0pt}
 	\item $1 \le lungime(P) \le 10 000$
	\item $1 \le lungime(CUV(i)) \le 50, unde 1 \le i \le N$
	\item $1 \le COST(i) \le 1000$, unde $1 \le i \le N$
	\item $1 \le N \le 100 000$
	\item Pentru $30 \%$ dintre teste: $1 \le lungime(P) \le 30, 1 \le N \le 50$
	\item Pentru $80 \%$ dintre teste: $1 \le lungime(P) \le 1000, 1 \le N \le 1000$
	\item Nu exist\u{a} cuvinte duplicate \^{i}n dic\c{t}ionar.
	\item $lungime(X)$ reprezint\u{a} lungimea \c{s}irului de caractere $X$.
	\item Toate caracterele sunt \^{i}n \^{i}ntervalul $'a'..'z'$.
\end{itemize}


\Example

\begin{example}
\exmp{
pemartesipeluna
7
pe 1
marte 2
si 2
luna 2
pe 1
sipe 4
pemarte 4
}{%
pe marte si pe luna
8
}%
\exmp{
anaaremere
8
ana 2
are 5
mere 3
arem 2
ar 1
em 2
er 1
e 1	
}{%
ana arem er e
6
}%
\end{example}

\Explanation

\^{I}n cazul primului exemplu: $1 + 2 + 2 + 1 + 2 = 8$. Descompunerea "pe marte sipe luna” are costul 1 + 2 + 4 + 2 = 9, etc...

Pentru cel de-al doilea exemplu: Costul 2 + 2 + 1 + 1 = 6 este minim. Descompunerea “ana are mere” are costul 2 + 5 + 3 = 10. Descompunerea “ana ar em er e” are costul 2 + 1 + 2 + 1 + 1 = 7.

PS: Da, dic\c{t}ionarul mayasilor nu prea e \^{i}n\c{t}eles de europeni… cu excep\c{t}ia lui Marinel.

\end{problem}
