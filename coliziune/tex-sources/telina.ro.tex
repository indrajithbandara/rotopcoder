\begin{problem}{\c{T}elin\u{a}}
{telina.in}{telina.out}
{0.5 secunde} {16 megabytes}{}

\c{T}elina, sau apium graveolens, este o plant\u{a} din familia apiaceae. Poate atinge o \^{i}n\u{a}l\c{t}ime de p\^{a}n\u{a} la 1 metru. Frunzele sale sunt mari, penat-lobate. Florile sunt mici, de obicei av\^{a}nd culoarea alb\u{a}. Fructul s\u{a}u este achen\u{a}. Poate rezista p\^{a}n\u{a} la temperaturi aproape de 0 grade Celsius. Perioada de \^{i}nflorire este la începutul toamnei. Este o plant\u{a} hidrofil\u{a}. [surs\u{a}: wikipedia.ro]

Se d\u{a} un \c{s}ir de $N$ caractere $S_1S_2 \dots S_N$. \c{T}elina cite\c{s}te (poate citi!) acest \c{s}ir din dreapta spre st\^{a}nga, \^{i}ncep\^{a}nd cu caracterul de pe pozi\c{t}ia N \c{s}i termin\^{a}nd cu cel de pe pozi\c{t}ia $1$.

Pe m\u{a}sur\u{a} ce aceasta parcurge caractere, le introduce \^{i}ntr-o stiv\u{a} $T$ (ini\c{t}ial vid\u{a}). \^{I}n momentul \^{i}n care cite\c{s}te caracterul $A_{N-i + 1}$ (la pasul $i$) \^{i}l introduce \^{i}n v\^{a}rful stivei $T$. Inainte de citi caracterul de la pasul $i+1$ \c{t}elina poate inversa toate caracterele din stiv\u{a} sau continu\u{a} la pasul urm\u{a}tor.

\c{T}elina v\u{a} roag\u{a} s\u{a} afla\c{t}i \c{s}irul minim lexicografic ce se poate forma \^{i}n stiva $T$ (primul element fiind in v\^{a}rful stivei iar ultimul element \^{i}n cap\u{a}tul stivei) la finalizarea opera\c{t}iilor de la pasul $N$.


\InputFile

\^{I}n fi\c{s}ierul de intrare $telina.in$ se afl\u{a} pe prima linie \c{s}irul $S$.

\OutputFile

\^{I}n fişierul de ie\c{s}ire $telina.out$ va con\c{t}ine pe prima linie \c{s}irul minim lexicografic ce se poate ob\c{t}ine \^{i}n $T$ aplic\^{a}nd opera\c{t}iile descrise anterior.

\Constraints
$1 \le N \le 10^{6}$.

$S$ con\c{t}ine doar caractere mici ale alfabetului englez.

\Example

\begin{example}
\exmp{
telina
}{%
anilet
}%
\end{example}

\Explanation

La pasul $1$, $T = a$. Dup\u{a} citirea caracterului $n$ stiva $T = na$. \^{I}nainte de a termina pasul $2$ \c{t}elina \^{i}ntoarce toate elementele din $T$ iar acum $T = an$.

Dup\u{a} cei $6$ pa\c{s}i $T = anilet$.

\end{problem}
